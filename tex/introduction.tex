\section{Introduction}
One of the most important components of power systems is the three-phase synchronous generator or alternator. Synchronous generators have two synchronously rotating fields. One field is produced by the rotor driven at synchronous speed and excited by DC current, and the other field is produced in the stator windings by the three-phase armature currents. The DC current for the rotor windings is produced is provided by the excitation systems. In older units, the exciters are DC generators mounted on the same shaft, providing excitation through slip rings. Today's systems use AC generators with rotating rectifiers, known as brushless excitation systems. The generator excitation system maintains generator voltage and controls the reactive power flow. Because they lack the commutator, AC generators can generate high power at high voltage, typically 30 kV. In a power plant, the size of generators can vary from 50 MW to 1500 MW. Steam turbines operate at relatively high speeds of 3600 or 1800 rpm for 60Hz operation. In a power system, several generators are operated in parallel in the power grid to provide the total power needed \cite{analysis}.

Loads of power systems are divided into industrial, commercial, and residential. Industrial loads are functions of voltage and frequency. They represent a large portion of system load due to the large number of induction motors that consume a large amount of reactive power. Commercial and residential loads consist largely of lighting, heating, and cooling. These loads are almost entirely resistive and consume extremely small amounts of reactive power. The magnitude of these three combined loads varies dramatically throughout the day. Power must be made available to consumers upon demand, therefore, an advanced control system in necessary to properly supply the amount of power needed by consumers \cite{analysis}.


